\documentclass[12pt]{article} % use larger type; default would be 10pt

\usepackage[utf8]{inputenc} % set input encoding (not needed with XeLaTeX)
%%% PAGE DIMENSIONS
\usepackage{geometry} % to change the page dimensions
\geometry{a4paper} % or letterpaper (US) or a5paper or....

\usepackage{csvsimple}
\usepackage{graphicx} 
\usepackage{appendix}
\usepackage{float}
\usepackage{pdfpages}
\usepackage{grffile}
\usepackage{tabulary}
\usepackage{booktabs} % for much better looking tables
\usepackage{paralist} % very flexible & customisable lists (eg. enumerate/itemize, etc.)
\usepackage{verbatim} % adds environment for commenting out blocks of text & for better verbatim
\usepackage{subfig} % make it possible to include more than one captioned figure/table in a single float
\usepackage{csvsimple}

%%% HEADERS & FOOTERS
\usepackage{fancyhdr} % This should be set AFTER setting up the page geometry
\pagestyle{fancy} % options: empty , plain , fancy
\renewcommand{\headrulewidth}{0pt} % customise the layout...
\lhead{}\chead{}\rhead{}
\lfoot{}\cfoot{\thepage}\rfoot{}

%%% SECTION TITLE APPEARANCE
\usepackage{sectsty}
\allsectionsfont{\sffamily\mdseries\upshape} % (See the fntguide.pdf for font help)
% (This matches ConTeXt defaults)

%%% ToC (table of contents) APPEARANCE
\usepackage[nottoc,notlof,notlot]{tocbibind} % Put the bibliography in the ToC
\usepackage[titles,subfigure]{tocloft} % Alter the style of the Table of Contents
\renewcommand{\cftsecfont}{\rmfamily\mdseries\upshape}
\renewcommand{\cftsecpagefont}{\rmfamily\mdseries\upshape} % No bold!

%%% END Article customizations
\usepackage{amssymb}
\usepackage{mathtools}
\usepackage{amsthm}
\newtheorem{theorem}{Theorem}[section]
%%% The "real" document content comes below...

\title{On Cycles of Generalized Collatz Sequences}
\author{\small Anant Gupta\\ 
    \small Lotus Valley International School, Noida, India\\
    \small \texttt{anantgupta35@gmail.com}}
\date{} 

%\setlength{\parindent}{0cm}
\begin{document}
\maketitle
\begin{abstract}

\end{abstract}

\section{Introduction}

\begin{equation}
F_k(n)= \begin{cases}    
    \frac{3n+k}{2}  & \text{if} \quad n \equiv 1 \mod 2\\
    \frac{n}{2}   & \text{if} \quad n \equiv 0 \mod2\
   \end{cases}
   \label{eqn:gcs}
\end{equation}


\section{Definitions}

We define the terms that will be used often in this paper. 
\begin{itemize}
    \item \textbf{Path} : Path of $F_k(n)$ is the set of numbers that are reached during successive iterations of the function. Path of $F_k(n) = \{F_k(n), F_k^2(n), F_k^3(n), \cdots \}$. For example, for $k=5$ and $n=12$, the subsequent numbers are $12 \rightarrow  6 \rightarrow  3\rightarrow 7 \rightarrow  13\rightarrow 22 \rightarrow  11\rightarrow 19\rightarrow 31\rightarrow  49\rightarrow 76 \rightarrow 38\rightarrow 19 \rightarrow  31\rightarrow \cdots$. Therefore the path of $F_5(12)$ is $\{12, 6, 3, 7, 13, 22, 11, 19, 31, 49, 76, 38, 19, 31\}$ and $12, 6 ..$ are elements of the path.
    
    \item \textbf{Cycle} : A cycle in $F_k$ is the list of values beginning from the minimum number that are reached repeatedly in the path of $F_k(n)$ for some $n \in \mathbb{N}$. We will refer to the minimum number in the cycle as $T_0$ in the rest of the paper. Thus, a cycle is $\{T_0, F_k(T_0), F^2_k(T_0), F^3_k(T_0), \cdots , F^s_k(T_0)\}$ with $F^{s+1}_k(T_0) = T_0$ and $s$ is the number of steps. For example, for $k=5$ and $n=12$ the path is $\{12 \rightarrow  6 \rightarrow  3\rightarrow 7 \rightarrow  13\rightarrow 22 \rightarrow  11\rightarrow 19\rightarrow 31\rightarrow  49\rightarrow 76 \rightarrow  38\rightarrow 19 \rightarrow  31\rightarrow 49 \rightarrow 76\rightarrow 38\rightarrow 19, \cdots\}$.  The cycle is $\{19, 31, 49, 76, 38\}$.
      
      \item \textbf{Set of Cycles $\zeta_k$} : We define $\zeta_k$ as a set of the minimum values of all the cycles in $F_k$. For example $\zeta_5 = \{ 1, 5, 19, 23, 187, 347\}$ as we have $6$ known cycles in $F_5$.
      
      \item \textbf{up iteration} : An up step is the usage of $\frac{3n+k}{2}$ once, $n$ being odd. An up iteration is the successive application of $\frac{3n+k}{2}$ due to a series of odd numbers in the path ending in an even number. The number of steps applied in the entire up iteration is represented by $u$ throught this paper.
      
      \item \textbf{down iteration} : A down step is the usage of $\frac{n}{2}$ once, $n$ being even. A down iteration is the successive application of $\frac{n}{2}$ due to a series of even numbers in the path. The number of steps applied in the entire down iteration is represented by $d$ throughout this paper.
      
      \item \textbf{orb} : An orb is defined as one up iteration and one down iteration.
      
      \item \textbf{2-cycle} : A 2-cycle is a set of $s$ orbs starting with an up iteration such that ending number is same as the starting number. For the rest of the paper we will represent a 2-cycle with $s$ orbs as $\{ u_1, d_1, u_2, d_2, \cdots , u_s, d_s \}$.  The $i^{th}$ up iteration step count is represented by $u_i$, while $i^{th}$ down iteration step count is represented by $d_i$. We will also interchangeably use $\{u_i, d_i\}$ to denote the full 2-cycle with $s$ orbs.
      
      \item \textbf{Original cycle of $F_k$} : A cycle $\{u_i, d_i\}$ is an original cycle of $F_k$ if $k$ is the smallest number where cycle appears in $F_k$ sequence.
      \item \textbf{Cycle inheritance}: A cycle is said to  be inherited from $F_r$ if there exists a cycle in $F_k$ and $F_r$ with identical sequence of $\{u_i,d_i\}$ where $k>r$.
      \item We would use the following definitions consistently throughout the paper
      \begin{align*}
            U  & = \sum_{1}^{s}u_i\\
            D  & = \sum_{1}^{s}d_i\\    
            \alpha_{i} & = 2^{\sum_{1}^{i-1}{u_j + d_j}} (3^{u_i} - 2^{u_i})3^{\sum_{i+1}^{s}u_j}\\
            \alpha & = \sum_{1}^{s} \alpha_{i}\\
            \beta  & = 2^{U + D} - 3^{U}\\ 
      \end{align*}
\end{itemize}

\section{Cycles in Generalized Collatz Sequence}

\begin{theorem}
Each cycle in $F_k$ for a given $k$ can be uniquely represented by its lowest element.
\end{theorem}
\begin{proof} 
Let each number be a node on a graph, then each node can only be directed towards one point, therefore the successive number for each node is fixed. Since there is only one path originating from a node, if a number is the lowest element of a cycle, it uniquely describes the cycle.
Therefore, all information about the cycles of $F_k$ is captured by $\zeta_k$.
\newline
\end{proof}
\textbf{Theorem 2 :} Let odd number $T_0$ be the starting point of an orb. The output of $F_k(n)$ after an up iteration with $u$ steps is 
\begin{equation*}
O_u = \frac{3^{u}T_0 +k(3^{u}-2^{u})}{2^{u}}.
\end{equation*}
\textbf{Proof :} Let the output be $O_1$ after one up step. Since $T_0$ is odd
\begin{equation*}
O_1 = \frac{3 T_0 + k}{2}
\end{equation*}
Therefore,the formula holds for $u=1$. Let the formula be true for $u$, then
\begin{equation*}
O_u=\frac{3^{u}T_0 +k(3^{u}-2^{u})}{2^{u}}
\end{equation*}
For $u+1$ up steps
\begin{align*}
O_{u+1} &=\frac{3\frac{3^{u}T_0 +k(3^{u}-2^{u})}{2^{u}}+k}{2}\\
&= \frac{3^{u+1}T_0 +k(3^{u+1}-2^{u}\cdot3+2^{u})}{2^{u+1}}\\
&=\frac{3^{u+1}T_0 +k(3^{u+1}-2^{u+1})}{2^{u+1}}\\
\end{align*}
Therefore, the formula is true for $u+1$ up steps. By mathematical induction, the formula is true for all $u \in \mathbb{N}$.
\newline

\textbf{Theorem 3:}
Let $T_0$ be the starting point of an orb. The output of $F_k(T_0)$ after s orbs with sequence $\{u_1, d_1, u_2, d_2, \cdots, u_s, d_s\}$ is
\begin{equation}
T_s = \frac{3^{U}T_0 + k\alpha}{2^{U+D}}
\label{eq:patheq}
\end{equation}

\textbf{Proof :}
Let $T_r$ be the output of $F_k$ after r orbs. From Theorem 2, output after $u_1$ up steps
\begin{equation*}
=\frac{3^{u_1}T_0 +k(3^{u_1}-2^{u_1})}{2^{u_1}}
\end{equation*}
and, the output $T_1$ after $1^{st}$ orb -  $u_1$ up steps followed by $d_1$ down steps is
\begin{align*}
T_1 & =\frac{\frac{3^{u_1}T_0 +k(3^{u_1}-2^{u_1})}{2^{u_1}}}{2^{d_1}}\\
    & =\frac{3^{u_1}n +k(3^{u_1}-2^{u_1})}{2^{u_1+d_1}}
\end{align*}
therefore, the formula holds for one orb.

Let the formula be true for a $s$ orbs, therefore 
\begin{equation*}
T_s = \frac{3^U T_0 + k \alpha}{2^{U+D}}
\end{equation*}

for $s+1$ orbs
\begin{align*}
T_{s+1} & =\frac{3^{u_s}\frac{3^{\sum u_i}T_0 + k(\sum_{i=1}^{s}(2^{\sum_{j=1}^{i-1}u_j+d_j})(3^{u_i}-2^{u_i})(3^{\sum_{j=i+1}^{s} u_j}))}{2^{\sum u_i+d_i}} +k(3^{u_s}-2^{u_s})}{2^{u_s+d_s}}\\
 & = \frac{3^{\sum u_i}T_0 + k(\sum_{i=1}^{s+1}(2^{\sum_{j=1}^{i-1}u_j+d_j})(3^{u_i}-2^{u_i})(3^{\sum_{j=i+1}^{s+1} u_j}))}{2^{\sum u_i+d_i}}\\
\end{align*}
therefore, the formula is true for $s+1$ orbs if it is true for $s$ orbs. By mathematical induction, the formula is true for all $s \in \mathbb{N}$.
\newline

\textbf{Theorem 4 :}
A trivial cycle  $\{k \rightarrow 2k\}$ exists for all $F_k$.

\textbf{Proof :}
Since $k$ is odd, path for $F_k(k)$ is $\frac{3k+k}{2}$ = $2k$. Since $2k$ is even, $F_k(2k) = \frac{2k}{2} = k$. Therefore, $k \rightarrow 2k \rightarrow k$ is always a cycle in $F_k$.
\newline

\textbf{Theorem 5 :} Minimal point $T_0$ and the orbs $\{u_i, d_i\}$ of cycles in $F_k$ satisfy 
\begin{equation}
    T_0 = \frac{k \alpha}{\beta}
    \label{eq:cycle}
\end{equation}

\textbf{Proof :}
Since $T_0$ is the minimal element of the cycle, the next number on the path from $T_0$ will be larger than $T_0$. Therefore, $T_0$ is odd. Given the path is a cycle, $T_s = T_0$, or from Theorem 3, 
\begin{align*}
T_0 & = T_s \\
    & = \frac{3^U T_0 + k \alpha}{2^{U+D}}\\
    & = \frac{k \alpha}{2^{U+D} - 3^U}\\
    & = \frac{k \alpha}{\beta}
\end{align*} 

We have assumed that $T_0$ is the the minimal element of the cycle. That assumption is actually not necessary. We can just assume that $T_0$ is the first odd number after an even number. This formulation will lead to multiple solutions for the same cycle with each orb's minimum point being a solution. Given each cycle will necessarily have at least one odd and one even element means every cycle will have an odd starting element.
\newline

\textbf{Theorem 6 :}
There is a cycle in $F_k$ \textit{iff} there is a solution to equation \eqref{eq:cycle} for some $\{u_i,d_i\}$.

\textbf{Proof :}
Theorem 5 shows that if there is a cycle, $T_0$ satisfies equation \eqref{eq:cycle}. The critical observation is that the formula additionally embeds all the the odd and even conditions of the collatz function. A path from $T_0$ can get back to $T_0$ only if at all iterations, correct odd/even choice is made and all elements on the path are integer. Even a single wrong odd / even path selection results in a fraction. Once a fraction is created, there is no way an integer can result through any number of application of $F_k$. 

First time a fraction is created from integer through incorrect path selection in $F_k$ will be either applying even path to odd number or applying odd path to even number. In either case, fraction $\frac{a}{2}$, where $a$ is an odd integer, results. Subsequent application of $F_k$ leaves similar fraction $\frac{a}{2^b}$, where $a$ is odd, $b > 0$. Both odd and even paths are considered below:
\begin{align*}
    F_k(\frac{a}{2^b})  & = \frac{3a/2^b + k}{2} \\
                        & = \frac{3a + k 2^b}{2^{b+1}}\\
                        & = \frac{c}{2^{b+1}}
\end{align*}
Note that $c$ is odd in above as $a$ is odd and $k \cdot 2^b$ is even, making the sum odd. As for even path, 
\begin{align*}
   F_k( \frac{a}{2^b}) & = \frac{a/2^b}{2}\\
                      & = \frac{a}{2^{b+1}}
\end{align*}

Theorem 6 has significant implications for collatz conjecture and we capture that in Theorem below.
\newline

\textbf{Theorem 7 :}
A non-trivial cycle in Collatz sequence exists if and only if for some $\{u_i, d_i\}$, $\beta \vert \alpha$. 

\textbf{Proof :}
From equation \eqref{eq:cycle}, the starting point of cycle in collatz sequence will be, 
\[ 
T_0 = \frac{\alpha}{\beta}.
\]
Given $T_0$ is integer, the only way this can happen is if $\beta  \vert \alpha$. Therefore, there is a non-trivial cycle if and only $\beta \vert \alpha$ for some $\{u_i, d_i\}$. Alternatively, collatz conjecture will be falsified if for any $\{u_i, d_i\}$, $\beta \vert \alpha$. Obviously, $\{u_1, d_1\} = \{1,1\}$ for trivial cycle satisfies this equation with $T_0 = 1$.
\newline

\textbf{Theorem 8 :}
Every cycle $\{u_i, d_i\}$ in $F_k$ will inherited by all $F_{rk}$ where $r \in \mathbb{N}$. The minimum element of the corresponding cycle in $F_{kr}$ will be $rT_0$, the cycle orbs in $F_{rk}$ be same as $\{u_i, d_i\}$ and each element in the cycle will be multiplied by $r$.

\textbf{Proof :}
Multiplying the cycle equation \eqref{eq:cycle} for cycle $\{u_i, d_i\}$ with $T_0$ starting point, by $r$
\[
r \cdot T_0 = \frac{(r k) \alpha}{\beta}
\]
From Theorem 6, $\{u_i, d_i\}$ is a cycle in $F_{rk}$ with starting point $r T_0$. It is also evident that all subsequent elements of the cycle in $F_{rk}$ are multiple of $r$.\\
\newline

\textbf{Theorem 9 :}
 For every number $L$, there exists a $k$ such that $F_k$ has more than $L$ cycles.
 
 \textbf{Proof :}
 Using Hardy and Ramanujan asymptotic partition function
 \[
 P(n) \asymp \frac{1}{4 n \sqrt{3}} e^{\pi \sqrt{2n/3}}
 \]
 we can find $n$ such that it has more than $L$ rotational invariant partitions. Set $U = D = n$,  and $k = 2^{U+D} - 3^U$. Equation \ref{eq:cycle} for $F_k$ reduces to 
\[T_0 = \alpha.\]
Therefore, every rotational invariant partition creates a cycle, yielding more than $L$ cycles.
\newline

\noindent
There are several consequences and observations based on the above theorems:
\begin{itemize}
    \item For a composite number $k = p_1 p_2 \cdots p_r$, the cycles of $F_k$ are union of cycles in all the factors of $k$ and the cycles that originate in $F_k$. For cycles that originate in $F_k$, $k$ divides $\beta$. For example, $F_{175}$ has $17$ non trivial cycles. It inherits $5$ cycles from $F_5$, $1$ cycle from $F_7$, $2$ original cycles of $F_{25}$, $2$ original cycles of $F_{35}$, and has $7$ of its own original cycles. Note that $5, 7$ being prime have only original non-trivial cycles and have $5$ and $1$ cycles, respectively. A cycle inherited in a multiple retains the $\{u_i, d_i\}$, whereas all the cycle elements are multiplied. In that sense, $\{u_i, d_i\}$ are invariant across the entire $F_k$ space and appear more fundamental than the cycle elements.   
    
    \item For all non-trivial cycles of $F_k$, atleast one of the factor of $k$ divides $\beta$ and if $k$ is prime, $k$ divides $\beta$.
    
    \item For every $\{u_i, d_i\}$, such that $\beta > 0$, the unique cycle originates in one and only one $F_k$, where
    \[ 
        k  = \frac{\beta}{gcd(\alpha, \beta)}.
    \]
    $\beta > 0$ is satisfied if $\frac{U \ln{3/2}}{\ln{2}}$. The unique mapping from $\{u_i, d_i\}$ to $k$ implies that cycles have a unique ordering. 
       
    \item If $k_1, k_2$ are co-primes, $F_{k_1}, F_{k_2}$ do not have any common non trivial cycle.
   
    \item Since $3^r$ does not divide $\beta$, all cycles of $k=3^r$ will require $\beta$ to divide $\alpha$. That is the same condition for non-trivial cycle in collatz sequence. This implies that all cycles of $F_{3^r}$ are exactly same as cycles of collatz sequence. Collatz conjecture can be proved or disproved for any $F_{3^r}$.
    
     
    \item All cycles of $F_k$ and $F_{3^p \cdot k}$ are same.
    
    \item All elements of path of $F_k(n k)$ will be multiple of $k$. Starting with $n k$, the next element in the path
\begin{align*}
F_k(n k) & =  
    \begin{cases} 
   \frac{3nk + k}{2} & = k \, \frac{3n + 1}{2} \quad  \text{if} \quad n \equiv 1 \mod 2\\
   \frac{nk}{2}  & = k \, \frac{n}{2}  \quad    \quad \: \,  \text{if} \quad n \equiv 0 \mod 2
   \end{cases}\\
   & = k \; F_k(n)
\end{align*}   
   
   \item For prime $k$ and $n$ not a multiple of $k$, path of $F_k(n)$ has no element that is multiple of $k$. This also implies that in such situation $F_k(n)$ will not converge to trivial cycle. 
   
   \item Assuming collatz conjecture to be true, all integers of the form $\frac{nk}{3^p}$ converge to trivial cycle in $F_k$. Conversely, if number is not of this form, it will not converge to trivial cycle.
   
   \item Assuming collatz conjecture to be true, only and all numbers of the form $\frac{nk}{3^{p}}$ converge to the trivial cycle, given an interval with a range that is an integral multiple of $k$, only $100\frac{3^p}{k}\%$ of the numbers will converge to the trivial cycle where $p$ is the maximum integral value for which $3^p$ divides $k$. 
    
    \item Under the assumption that $F_k$ converges, every $k$ that is not a multiple of $3$ will have atleast one original cycle.
\end{itemize}
 
\begin{thebibliography}{9}

\bibitem{Belaga98}
Edward G. Belaga, Maurice Mignotte.
\textit{Embedding the 3x+1 Conjecture in a 3x+d Context}
Experimental Mathematica 7:2, 145-151.

\bibitem{eliahou}
Eliahou, Shalom
\textit{The 3x + 1 problem: new lower bounds on nontrivial cycle lengths}
Discrete Mathematics. 118 (1): 45–56. doi:10.1016/0012-365X(93)90052-U.

\bibitem{eric}
Eric Roosendaal.
\textit{On the 3x + 1 problem}
http://www.ericr.nl/wondrous/

\bibitem{Hayden}
Messerman, Hayden; LeBeau, Joey; Klyve, Dominic.
\textit{Generalized Collatz Functions: Cycle Lengths and Statistics}
International Journal of Undergraduate Research and Creative Activities, Volume 4, Article 2, 2012/10/09

\bibitem{john2004}
John Lesieutre.
\textit{On a Generalization of the Collatz Conjecture}
https://www.semanticscholar.org/paper/On-a-Generalization-of-the-Collatz-Conjecture-John-Wang/9da2c36caa6a8aca9ed0c3a03b3fe0efd32eb2a8

\bibitem{kurtz2007}
Kurtz, Stuart A.; Simon, Janos (2007). 
\textit{The Undecidability of the Generalized Collatz Problem}
In Cai, J.-Y.; Cooper, S. B.; Zhu, H. (eds.). Proceedings of the 4th International Conference on Theory and Applications of Models of Computation, TAMC 2007, held in Shanghai, China in May 2007. pp. 542–553. doi:10.1007/978-3-540-72504-6-49. ISBN 978-3-540-72503-9. As PDF

\bibitem{lagaris85}
Lagarias, Jeffrey C.
\textit{The 3x + 1 problem and its generalizations}
The American Mathematical Monthly. 92 (1): 3–23. doi:10.1080/00029890.1985.11971528. JSTOR 2322189.

\bibitem{lagaris90}
Lagarias, Jeffrey C.
\textit{The Set of Rational Cycles for the 3x+1 Problem}
Acta Arith. 56, 33-53, 1990.

\bibitem{lagarisbook}
Lagarias, Jeffrey C.
\textit{The ultimate challenge: the 3x + 1 problem}
Providence, R.I.: American Mathematical Society. p. 4. ISBN 978-0821849408.

\bibitem{simon2005}
Simons, John L. 
\textit{On the nonexistence of 2-cycles for the 3x + 1 problem}
Math. Comp. 74: 1565–72. Bibcode:2005MaCom..74.1565S. doi:10.1090/s0025-5718-04-01728-4. MR 2137019.
 
 \bibitem{simon2003}
 Simons, J.; de Weger, B. 
 \textit{Theoretical and computational bounds for m-cycles of the 3n + 1 problem}
 Acta Arithmetica. 117 (1): 51–70. Bibcode:2005AcAri.117...51S. doi:10.4064/aa117-1-3.

\bibitem{steiner}
Steiner, R. P. 
\textit{A theorem on the syracuse problem}
Proceedings of the 7th Manitoba Conference on Numerical Mathematics. pp. 553–9. MR 0535032.
 
\bibitem{terrance}
Tao, Terence.
\textit{Almost all Collatz orbits attain almost bounded values}
https://arxiv.org/abs/1909.03562 
\end{thebibliography}

\end{document}
